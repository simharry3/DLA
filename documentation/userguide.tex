%%%%%%%%%%%%%%%%%%%%%%%%%%%%%%%%%%%%%%%%%
% Stylish Article
% LaTeX Template
% Version 2.1 (1/10/15)
%
% This template has been downloaded from:
% http://www.LaTeXTemplates.com
%
% Original author:
% Mathias Legrand (legrand.mathias@gmail.com) 
% With extensive modifications by:
% Vel (vel@latextemplates.com)
%
% License:
% CC BY-NC-SA 3.0 (http://creativecommons.org/licenses/by-nc-sa/3.0/)
%
%%%%%%%%%%%%%%%%%%%%%%%%%%%%%%%%%%%%%%%%%

%----------------------------------------------------------------------------------------
%	PACKAGES AND OTHER DOCUMENT CONFIGURATIONS
%----------------------------------------------------------------------------------------

\documentclass[fleqn,10pt]{UserGuideArx} % Document font size and equations flushed left

\usepackage[T1]{fontenc}

% \usepackage{microtype}

\usepackage[english]{babel} % Specify a different language here - english by default

\usepackage{lipsum} % Required to insert dummy text. To be removed otherwise

%----------------------------------------------------------------------------------------
%	COLUMNS
%----------------------------------------------------------------------------------------

\setlength{\columnsep}{0.55cm} % Distance between the two columns of text
\setlength{\fboxrule}{0.75pt} % Width of the border around the abstract

%----------------------------------------------------------------------------------------
%	COLORS
%----------------------------------------------------------------------------------------

\definecolor{color1}{RGB}{0,0,90} % Color of the article title and sections
\definecolor{color2}{RGB}{0,20,20} % Color of the boxes behind the abstract and headings

%----------------------------------------------------------------------------------------
%	HYPERLINKS
%----------------------------------------------------------------------------------------

\usepackage{hyperref} % Required for hyperlinks
\hypersetup{hidelinks,colorlinks,breaklinks=true,urlcolor=color2,citecolor=color1,linkcolor=color1,bookmarksopen=false,pdftitle={Title},pdfauthor={Author}}

%----------------------------------------------------------------------------------------
%	ARTICLE INFORMATION
%----------------------------------------------------------------------------------------

\JournalInfo{Journal, Vol. XXI, No. 1, 1-5, 2013} % Journal information
\Archive{Additional note} % Additional notes (e.g. copyright, DOI, review/research article)

\PaperTitle{DIGLS User's Guide} % Article title

\Authors{Clayton Rayment\textsuperscript{1}} % Authors
% \affiliation{\textsuperscript{1}\textit{Department of Biology, University of Examples, London, United Kingdom}} % Author affiliation
% \affiliation{\textsuperscript{2}\textit{Department of Chemistry, University of Examples, London, United Kingdom}} % Author affiliation
% \affiliation{*\textbf{Corresponding author}: john@smith.com} % Corresponding author

\Keywords{DLA --- Visualization --- OpenGL} % Keywords - if you don't want any simply remove all the text between the curly brackets
\newcommand{\keywordname}{Keywords} % Defines the keywords heading name

%----------------------------------------------------------------------------------------
%	ABSTRACT
%----------------------------------------------------------------------------------------

\Abstract{DIGLS (DLA Interactive GL Simulator) is an efficient single-threaded application for performing Diffusion Limited Aggregation simulations, along with a companion OpenGL visualizer using the freeGLUT GL Utility Toolkit. This doccument serves as a a guide for installation and operation of DIGL. Code for this project is available here: \url{https://github.com/simharry3/DLA}}

%----------------------------------------------------------------------------------------

\begin{document}

\flushbottom % Makes all text pages the same height

\maketitle % Print the title and abstract box

\tableofcontents % Print the contents section

\thispagestyle{empty} % Removes page numbering from the first page

%----------------------------------------------------------------------------------------
%	ARTICLE CONTENTS
%----------------------------------------------------------------------------------------

% \section*{Introduction} % The \section*{} command stops section numbering

% \addcontentsline{toc}{section}{Introduction} % Adds this section to the table of contents

%------------------------------------------------

\section{Features}

\subsection{Graphics Modes}
DIGILS features two different graphics modes: Fast and Fancy. Fast graphics renders all particles as points. Aggregator particles are shown in red, and Active particles are shown in green. Fast graphics will render each particle as a sphere, complete with local lighting effects, however visualizer performance will take a significant hit with large numbers of particles.
\subsection{Bounding Box}
By default, a bounding box is drawn around the simulation space, in order to distinguish the simulation volume from the background. Should the user desire, this bounding box may be toggled.

\subsection{Rotation}
By default, DIGILS will rotate the camera around the simulation space as the simulation progresses. Should the user desire, this rotation can be toggled on and off.

\subsection{Active Particle Display}
Active particles are not displayed by default, as this can greatly reduce visualizer performance at the beginning of a simulation should the number of particles be large. The user however may toggle the display of active particles in both Fast and Fancy graphics modes.
\subsection{Morton Line Display}
With the most recent release of DIGILS, the user may toggle on or off the morton line which the simulation uses to calculate collisions. More details about this can be found in the paper titled "Visualization of Diffusion Limited Aggregation Simulations using OpenGL".
\subsection{Zoom}
Using the mouse wheel, the user may adjust the position of the camera closer or farther to the simulation center (which is calculated automatically at the beginning of visualization).
\section{Operation}
\subsection{Aggregator Generation}
An aggregator file is a space-seperated value list containing only numbers in the following format:
\begin{center}
    \texttt{<X> <Y> <Z>}
\end{center}
Each line in the file represents a new aggregator particle in the system. Several sample aggregate structures are located in the \texttt{<dir>/samples} directory. Should the user not specify an imput aggregator file a default aggregatior will be placed at the center of the simulation.
\subsection{Build Simulation}
Cmake files are included to automate the build process. From your build directory simply run
\begin{center}
    \texttt{cmake <dir>}
\end{center}
where \texttt{<dir>} is the path to the main \texttt{DLA/} directory which contains \texttt{CMakeLists.txt}.
Then, from your build directory, simply run:
\begin{center}
    \texttt{make}
\end{center}
You will now find the executables in the \texttt{<dir>/bin/} directory.
\subsection{Run Simulation}
Once the project has been built, the simulation can be run from the command line using:
\begin{center}
    \texttt{<dir>/bin/simulation <Size> <Particles>}
\end{center}
A third, optional argument may be used to specify the location of the input aggregator file, otherwise the default aggregation will be used.
\subsection{Visualizer}
Several options are available to the user from the visualizer, as displayed at the bottom of the visualization window:
\begin{description}
    \item[Esc] Close the simulation window and end the simulation.
    \item[F1] Toggle render mode between fast and fancy.\\ (Default: fast)
    \item[F2] Toggle bounding box to cover the simulation space.\\ (Default: off)
    \item[F3] Toggle rotation of camera around simulation.\\ (Default: off)
    \item[F4] Toggle visibility of active particles in simulation.\\ (Default: on)
    \item[F5] Toggle visibility of Morton encoding line.\\ (Default: off)
    \item[Scroll] Move camera closer to or farther from simulation.\\
\end{description}
%------------------------------------------------



%------------------------------------------------
\phantomsection
% \section*{Acknowledgments} % The \section*{} command stops section numbering
% Morton Encoding code taken from [1]
% %----------------------------------------------------------------------------------------
%	REFERENCE LIST
% %----------------------------------------------------------------------------------------
% \phantomsection
% \bibliographystyle{unsrt}
% \bibliography{references}

%----------------------------------------------------------------------------------------

\end{document}